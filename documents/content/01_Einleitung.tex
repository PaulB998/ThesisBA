\chapter{Einleitung}

\section{Motivation}
In jedem Bereich eines Unternehmens entstehen täglich umfangreiche und
wertvolle Datensätze. Das Erfassen, Analysieren und Nutzen dieser Daten bietet
\acro{API} großes Potenzial, um Unternehmensprozesse zu optimieren und sowohl
die Transparenz als auch den Wissensaustausch zu verbessern. Eine einfache und
schnelle Suche, um spezifische Informationen aus diesem Datensatz in
aufbereiteter Form als Antwort zu bekommen, hat einen enormen Mehrwert für das
Unternehmen. Z.B. kann ein bestimmtes Skill-Set mit einer einfachen Suche
\acro{API} gefunden werden. Eine Möglichkeit diese Daten schnell und in
aufbereiteter Form bereitzustellen, ist die Nutzung von KI-Modellen. In vielen
Unternehmen, die Microsoft-Tools verwenden, wird beispielsweise bereits der
Copilot genutzt oder befindet sich in der Evaluationsphase. Durch den Zugriff
auf unternehmensinterne Daten kann das KI-System spezifischere Informationen
verarbeiten, wodurch die Interaktion präziser und auf das Unternehmen
zugeschnittener wird. Dabei ist nicht nur die Datenbasis ein wichtiger
Bestandteil für qualitativ hochwertige Antworten, sondern auch die Eingabe des
Nutzers, bekannt unter dem Begriff „Prompt Engineering“. Die Anwendungsfälle,
die im Unternehmen durch neue KI-Systeme aufkommen oder unterstützter werden,
variieren von Unternehmen zu Unternehmen. Deshalb ist es besonders wichtig
diese Anwendungsfälle zu identifizieren, analysieren und die Möglichkeiten der
Umsetzung mit den aktuellen KI-Tools zu erforschen. Der Mehrwert des Microsoft
Copilot in Unternehmen ist bereits erkennbar. Eine Umfrage zeigt, dass 75
\cite{kalliamvakouResearchQuantifyingGitHub2022} prozent der Nutzer Zeit bei
der Suche in eigenen Daten einsparen. Darüber hinaus nutzen 73 prozent der
Befragten den Copilot zur Erstellung von Erstentwürfen. Zusätzlich berichten 68
prozent der Nutzer, dass die Qualität ihrer Arbeit durch den Einsatz des
Copilot verbessert wird . Schon nach ca. einem Jahr zeigt sich der Mehrwert des
Microsoft Copilot in Unternehmen und ist aus dem Alltag der Nutzer nicht mehr
wegzudenken.

\section{Erwartete Ergebnisse}

\section{Vorgehen}

\section{Abgrenzung}