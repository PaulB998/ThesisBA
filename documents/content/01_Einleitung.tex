\chapter{Einleitung}

\section{Motivation}

Der Bedarf an automatisierten Lösungen zur Unterstützung der täglichen Arbeitsabläufe wächst 
kontinuierlich \footcite{demary_kunstliche_2019} \footcite{feike_kunstliche_2024}. In vielen Unternehmensbereichen kann der Einsatz von \ac{KI} zur Prozessoptimierung erheblich zur Effizienzsteigerung und Fehlerreduktion beitragen. Solche Lösungen sind besonders 
wertvoll in komplexen Arbeitsumgebungen, wo herkömmliche manuelle Methoden zunehmend an ihre 
Grenzen stoßen \footcite{lenzen_kunstliche_2023}.
Das Setting bei ObjektKultur ist geprägt von einer Vielzahl an Dokumenten und Modulen innerhalb der 
Plattform. Diese Module decken verschiedene Bereiche wie Projektmanagement, Ressourcenplanung 
und Dokumentenverwaltung ab. Angesichts der großen Menge an Daten und der Vielzahl von Aufgaben 
kann die Suche nach relevanten Informationen zeitaufwendig sein. Ein \ac{KI}-gestützter Chatbot könnte 
hier erhebliche Vorteile bieten, indem er den Zugriff auf Informationen beschleunigt und die Benutzer 
in Echtzeit unterstützt. Dies würde die Effizienz der Arbeitsabläufe deutlich erhöhen und die Mitarbeiter 
entlasten.
Die OKPlattform dient als zentrale Anlaufstelle für die Verwaltung von Projekten, Ressourcen und 
Dokumenten in einem Unternehmen. Die Anforderungen an Effizienz und Qualität bei der Verwaltung 
dieser Prozesse sind kontinuierlich gestiegen. Dies hat zur Folge, dass herkömmliche manuelle 
Methoden zur Bewältigung von Aufgaben zunehmend an ihre Grenzen stoßen.
In diesem Kontext stellen die Fragen zur Kapazitätsplanung, Rechnungsstellung und Auftragsverwaltung 
zentrale Herausforderungen dar. Die steigende Anzahl an Aufgaben und die Notwendigkeit schneller 
und präziser Antworten auf Budgetierungsfragen sowie Einschätzungen zur Mitarbeiterkapazität 
(beispielsweise im laufenden Sprint) erfordern effiziente Hilfsmittel. Fragen wie „Wie lange reicht unser 
Budget noch unter den aktuellen Voraussetzungen?“ oder „Wie viel Kapazität haben wir für den 
nächsten Sprint?“ können mit den derzeitigen Methoden nur ineffizient beantwortet werden.
Ein \ac{KI}-Assistent könnte hierbei helfen, indem er aktuelle Daten analysiert und Prognosen erstellt, die 
den Mitarbeitern als Entscheidungsgrundlage dienen. Durch die Automatisierung dieser 
Routineaufgaben und die Beschleunigung von Informationsabrufen könnte die Effizienz der 
Arbeitsprozesse in der OKPlattform erheblich gesteigert werden. Der KI-Assistent soll schnell auf 
Anfragen reagieren, indem er relevante Daten auswertet und die notwendigen Informationen 
bereitstellt. Dies würde die Mitarbeiter entlasten und es ihnen ermöglichen, sich auf komplexere und 
strategisch wichtigere Aufgaben zu konzentrieren.
Als Teil der genannten Prozessschritte spielt die Dokumentenauswertung eine besondere Rolle. Die 
Dokumente, die in der Projektverwaltung der OKPlattform (\ac{PQC}) gespeichert sind, enthalten 
wesentliche Informationen zu laufenden und vergangenen Projekten. Zudem sind 
Vertragsbedingungen, NDAs und AGBs im Dateibereich des MS Teams zu den jeweiligen Projekten 
hinterlegt. Die manuelle Durchsicht und Auswertung dieser Dokumente ist zeitaufwendig und 
fehleranfällig. Durch die Automatisierung dieser Routineaufgaben und die Beschleunigung von 
Informationsabrufen könnte die Effizienz der Arbeitsprozesse in der OKPlattform erheblich gesteigert 
werden. Der \ac{KI}-Assistent soll schnell auf Anfragen reagieren, indem er relevante Daten auswertet und 
die notwendigen Informationen bereitstellt. Dies würde die Mitarbeiter entlasten und es ihnen 
ermöglichen, sich auf komplexere und strategisch wichtigere Aufgaben zu konzentrieren.
Die Prüfung der Hypothesen zum Einsatz von KI-Tools in den genannten Prozessen ist daher Gegenstand 
dieser Thesis. Es muss untersucht werden, inwiefern eine KI diese entsprechenden Anforderungen 
tatsächlich erfüllen kann.

\section{Zielsetzung}

Das Hauptziel dieser Arbeit ist die Entwicklung eines prototypischen, \ac{KI}-gestützten Assistenzsystems zur 
Optimierung der Arbeitsprozesse bei einem mittelständischen Softwarehersteller. Durch die 
Automatisierung von Routineaufgaben sowie die Beschleunigung von Informationsabrufen soll die 
Effizienz der Arbeitsabläufe gesteigert werden.
Ein zentrales Ziel ist die Verbesserung der Effizienz und Qualität der Arbeitsprozesse. Das System soll in 
der Lage sein, relevante Informationen aus verschiedenen Dokumenten zu extrahieren und den Nutzern 
zur Verfügung zu stellen, sowie Fragen zur Kapazitätsplanung, Rechnungsstellung und Projektdaten zu 
beantworten. Es soll den Nutzern helfen, ihre Ressourcen effizient zu planen und den Zugriff auf 
historische Projektdaten zu erleichtern. Somit soll das KI-gestützte Assistenzsystem zu einem noch 
effizienteren und nutzerfreundlicheren Werkzeug für die Projekt- und Ressourcenverwaltung werden.
Zunächst werden die Anforderungen und Rahmenbedingungen für das zu entwickelnde System 
spezifiziert. Dies beinhaltet die Definition der spezifischen Anforderungen, die das System erfüllen muss, 
um den Nutzern bei der Planung ihrer Ressourcen und dem Zugriff auf historische Projektdaten zu 
helfen.
Anschließend erfolgt die Konzeption des Prototyps. In diesem Schritt werden verschiedene Lösungen, 
Architekturen und Implementierungsmöglichkeiten untersucht. Darüber hinaus wird geprüft, wie das 
System in die bestehende Infrastruktur der OKPlattform integriert werden kann.
Daraufhin wird der Prototyp umgesetzt. Das resultierende System soll die Nutzer der OKPlattform in 
ihren täglichen Arbeitsaufgaben unterstützen, indem es Routineaufgaben vereinfacht und schnelle, 
präzise Antworten auf diverse Fragen liefert.
Abschließend wird der Prototyp evaluiert. Diese Evaluation umfasst die Messung der Vorteile des 
Systems und soll sicherstellen, dass das \ac{KI}-gestützte Assistenzsystem zu einem noch effizienteren und 
nutzerfreundlicheren Werkzeug für die Projekt- und Ressourcenverwaltung wird.
Die Prüfung der Hypothesen zum Einsatz von \ac{KI}-Tools in den genannten Prozessen ist daher Gegenstand 
dieser Thesis. Es muss untersucht werden, inwiefern eine \ac{KI} diese entsprechenden Anforderungen 
tatsächlich erfüllen kann, ohne an Grenzen zu stoßen.

\section{Vorgehen}

Die Grundlagen dieser Thesis basieren auf einer umfassenden theoretischen Recherche. Hierbei werden 
relevante Fachbücher, Fachzeitschriften und wissenschaftliche Artikel herangezogen. Ergänzend 
werden Internet-quellen verwendet, um den aktuellen Stand der Technik und die neuesten 
Entwicklungen im Bereich der \ac{KI}-gestützten Assistenzsysteme und \ac{KI}-Tools zu erfassen. Dies dient der 
Lösungskonzeption und der Berücksichtigung des State of the Art bei der Entwicklung des 
Prototyps.
Ein wichtiger Bestandteil des Vorgehens ist die Analyse bestehender Lösungen und Architekturen für 
ähnliche Systeme. Dies umfasst die Evaluierung von Technologien wie Azure OpenAI und anderen 
relevanten Tools zur Entwicklung eines solchen Assistenzsystems. Durch eine detaillierte Untersuchung 
dieser Technologien sollen die bestmöglichen Ansätze für die Implementierung in der OKPlattform 
identifiziert werden. Diese Phase umfasst die Konzeption und Implementierung unter Berücksichtigung 
von Best Practices.
Zur Ergänzung der theoretischen Grundlagen wird eine qualitativ empirische Erhebung durchgeführt. 
Hierbei werden Interviews mit Mitarbeitern der OKPlattform geführt. Diese Interviews dienen dazu, 
spezifische Anforderungen und Herausforderungen aus der Praxis zu identifizieren und die 
theoretischen Erkenntnisse zu validieren. Die Ergebnisse dieser Interviews fließen in die Konzeption des 
Systems ein.
Des Weiteren wird ein prototypischer Ansatz verfolgt. Anhand der gewonnenen Erkenntnisse wird ein 
Prototyp des \ac{KI}-gestützten Assistenzsystems entwickelt und implementiert. Hierbei wird ein agiles 
Vorgehen gewählt, das einen kontinuierlichen Austausch mit den „Kunden“ sowie eine kontinuierliche 
Evaluation beinhaltet. Durch diesen praktischen Ansatz sollen die Funktionalitäten und der Mehrwert 
des Systems überprüft und gegebenenfalls optimiert werden.
Abschließend erfolgt die Evaluation des Prototyps. Die Evaluation umfasst die Messung der Vorteile des 
Systems, um sicherzustellen, dass das \ac{KI}-gestützte Assistenzsystem zu einem noch effizienteren und 
nutzerfreundlicheren Werkzeug für die Projekt- und Ressourcenverwaltung wird.

\section{Abgrenzung}

Die vorliegende Arbeit verfolgt das Ziel, ein prototypisches KI-gestütztes Assistenzsystem zur Optimierung von Prozessen in der OKPlattform zu entwickeln. Die angestrebte Optimierung zielt auf eine Steigerung der Effizienz und Genauigkeit, insbesondere durch die Automatisierung von Routineaufgaben wie der Dokumentenauswertung, der \ac{PQC}, der Kapazitätsplanung sowie der Rechnungsstellung. Der Schwerpunkt liegt dabei auf der Analyse und Beantwortung von Fragen sowie der Bewertung von Dokumenten.

Die vorliegende Arbeit dient als Machbarkeitsstudie sowie als Referenz für die Integration von KI in bestehende Unternehmensprozesse. Der Fokus liegt nicht auf der Entwicklung eines umfassenden Systems für alle denkbaren Anwendungsfälle, sondern auf der Umsetzung und Evaluierung grundlegender Konzepte innerhalb spezifischer, praxisnaher Anwendungsfälle. Die gewonnenen Erkenntnisse bilden die Grundlage für zukünftige Erweiterungen und spezifische Anpassungen in größeren Projekten.

Das Ziel besteht nicht in einer vollständigen Automatisierung aller Prozesse der OKPlattform, sondern in der exemplarischen Erarbeitung grundlegender Ansätze zur Optimierung, die auf verschiedene Bereiche übertragen werden können.