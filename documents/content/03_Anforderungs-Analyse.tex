\chapter{Anforderungs-Analyse}

\section{IST-Zustand}

Die OKPlattform stellt eine inhouse entwickelte webbasierte Software dar, welche spezifisch auf die Bedürfnisse und Anforderungen des Unternehmens zugeschnitten ist. Die Entwicklung der OKPlattform zielte darauf ab, verschiedene Geschäftsprozesse abzubilden. Zudem dient sie als Rahmenanwendung und zentrales System, welches eine Vielzahl von Modulen und Funktionen umfasst, darunter Buchhaltung, PQC, Kapazitätsplanung, Rechnungsstellung, Skills und Auftragsverwaltung.

Die Eigenentwicklung der OKPlattform ermöglicht eine exakte Abstimmung auf die spezifischen Anforderungen und Arbeitsabläufe des Unternehmens. Dies resultiert in einer optimalen Anpassung an die vorhandenen Strukturen und Prozesse, was die Benutzerfreundlichkeit und Effizienz erheblich steigert. Zudem kann die Plattform flexibel erweitert und angepasst werden, um auf sich ändernde Bedürfnisse und neue Herausforderungen im Unternehmen zu reagieren.

\subsection{PQC}

Das Project Quality Controlling (PQC) stellt ein Instrument dar, welches primär von Projektleitern genutzt wird, um die Qualität und den Fortschritt von Projekten zu überwachen. Es bildet den Controlling-Prozess für Projekte bei Objektkultur ab.
Derzeit umfasst das PQC diverse Funktionen, darunter die Führung von Projektsteckbriefen, Scope-Protokollen sowie Qualitätsbewertungen.

\subsection{Rechnungsstellung}

Die Rechnungsstellung innerhalb der OKPlattform umfasst sämtliche Prozesse, die mit der Erstellung von Rechnungen in Verbindung stehen. Derzeit erfolgt die Rechnungsstellung durch die Eingabe von Daten sowie eine anschließende Überprüfung der tatsächlichen PT mit den Soll-PT.

\subsection{Auftragsverwaltung/Kapazitätsplannung}

Die OKPlattform dient der Auftragsverwaltung und Kapazitätsplanung, wobei sie für die Organisation und Zuteilung von Kapazitäten der Mitarbeiter innerhalb von Projekten verantwortlich ist. Die Kapazitätsplanung erfolgt derzeit überwiegend manuell, wobei auch Urlaube eingetragen werden.

\subsection{Dokumente}

Die in Dokumenten enthaltenen Informationen zu Projekten, Verträgen und weiteren relevanten Inhalten sind für die Arbeit mit diesen Dokumenten von großem Nutzen. Allerdings müssen sie in der Regel manuell auf verschiedenen Plattformen gesucht werden, was einen hohen Zeitaufwand bedeutet.

\subsubsection{Sharepoint}

Die Funktionalität von SharePoint umfasst die zentrale Ablage und Verwaltung von Dokumenten sowie die Bereitstellung einer Intranet-Plattform für Unternehmen. Das Intranet ermöglicht eine nahtlose Integration verschiedener Informationsquellen und Arbeitsprozesse, wodurch eine Optimierung der Kommunikation und Zusammenarbeit innerhalb des Unternehmens erzielt wird. Obgleich SharePoint kein integraler Bestandteil der inhouse entwickelten OKP-Plattform ist, besteht eine enge Verknüpfung zwischen beiden, sodass SharePoint eine wesentliche Rolle bei der Bereitstellung und Organisation von Dokumenten spielt, welche in verschiedenen Projekten Verwendung finden. Eine wesentliche Herausforderung bei der Nutzung von SharePoint besteht in der effizienten Suche nach spezifischen Informationen, insbesondere in großen und komplexen Dokumentenbeständen. Ein intelligentes Auswertungssystem könnte diese Herausforderung adressieren, indem es den Zugriff auf relevante Inhalte automatisiert und vereinfacht.

\subsubsection{OneDrive}

Die primäre Nutzung von OneDrive innerhalb des Unternehmens besteht in der persönlichen Speicherung und dem Austausch von Dateien. Analog zu SharePoint ist auch OneDrive kein direktes Modul der OKP-Plattform, sondern ein integraler Bestandteil der von Microsoft bereitgestellten Infrastruktur, welche das Unternehmen nutzt. Die Herausforderung bei der Verwendung von OneDrive liegt in der Vielzahl der gespeicherten Dokumente, die ohne eine systematische Kategorisierung schwer zu durchforsten sind. Die Implementierung eines intelligenten Auswertungssystems, welches auf diese Dateien zugreifen kann, würde es den Nutzern ermöglichen, relevante Informationen schneller zu finden und effizienter zu arbeiten.

\subsubsection{Teams}

Die Kommunikations- und Kooperationsplattform Microsoft Teams wird bei ObjektKultur eingesetzt. Die Speicherung von in Teams geteilten Dateien erfolgt entweder in SharePoint (für Kanäle) oder OneDrive (für private Chats). Dateien im PQC werden häufig mit dem Label "teams.common.attachment.download.label" angezeigt. Diese Struktur erleichtert die Integration von Dateien in die tägliche Kommunikation, kann jedoch die Auffindbarkeit erschweren, wenn Dokumente über viele Kanäle verstreut sind. 

\subsection{Reporting(PowerBI)}

//TODO
Zu klären: realisierbar?

\section{Soll-Analyse}

\subsection{Experteninterviews}

Im Rahmen dieser Arbeit wurden Experteninterviews durchgeführt, um die Anforderungen für ein KI-Tool zu erheben. Diese qualitative Forschungsmethode ermöglichte es, tiefgehende Einblicke in die Prozesse und Vorgehensweisen der Nutzer der OKPlattform zu gewinnen. Dazu wurden die Erfahrungen der aktiven Nutzer gezielt abgefragt.\footcite[4]{wernitz_experteninterview_2018}

\subsubsection{Zielsetzung}

Die Zielsetzung der durchgeführten Interviews bestand in der Ermittlung der Anforderungen an ein KI-Assistenztool, welches die Beschaffung von Informationen erleichtern soll. Dabei wurden sowohl häufige als auch unklare Informationsbeschaffungswege berücksichtigt. Die Befragungen dienten der Erfassung bestehender Nutzungsmuster, potenzieller Herausforderungen sowie Vorschläge und Erwartungen der Nutzerinnen und Nutzer. Diese Anforderungen bilden die Grundlage für die Umsetzung des Prototyps.

\subsubsection{Auswahl der Experten}

Die Auswahl der Experten erfolgte zunächst anhand der Nutzungsfrequenz der OKPlattform. Hierbei wurden insbesondere Personen berücksichtigt, die regelmäßig mit den Modulen \ac{PQC}, Rechnungsstellung und Kapazitätsplanung arbeiten. Die Auswahl basiert auf praktischer Erfahrung, um sicherzustellen, dass die Anforderungen praxisrelevant und umsetzbar sind. \footcite[5]{wernitz_experteninterview_2018}

\subsubsection{Methodik und Durchführung}

Die durchgeführten Interviews waren halbstrukturiert, um einerseits zielgerichtete Fragen zu stellen und andererseits einen offenen Dialog zuzulassen, in dessen Verlauf alle relevanten Informationen extrahiert werden können.

Die Schwerpunkte der Interviews lassen sich wie folgt zusammenfassen:

\begin{itemize}
    \item Tägliche Nutzung der OKPlattform und der genutzten Module
    \item Identifikation von Schwachstellen und Herausforderungen
    \item Anforderungen an ein System, das die Informationsbeschaffung durch KI unterstützt
\end{itemize}

\subsection{Use-Cases}