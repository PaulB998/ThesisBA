\chapter{Grundlagen}

\section{KI}

\subsection{Überblick Künstliche Intelligenz}

In den vergangenen Jahren hat die Künstliche Intelligenz (KI) zunehmend an Bedeutung gewonnen. Siehe dazu: OECD (2024), S. 28–30. Der Begriff bezeichnet die Fähigkeit, kognitive Funktionen auszuführen, die ansonsten der menschlichen Intelligenz zugeordnet werden. Zu den Fähigkeiten, die darunter subsumiert werden, zählen das Verstehen natürlicher Sprache, die Bilderkennung, das Treffen von Entscheidungen sowie das Lösen komplexer Probleme. Des Weiteren soll sie aus Erfahrung lernen, Muster erkennen und Aufgaben eigenständig erledigen können, ohne dafür explizit programmiert werden zu müssen. \footcite[9]{gethmann_kunstliche_2022}

\ac{KI} hat viele Anwendungsbereiche und reicht in unterschiedliche Sektoren in der Industrie über die Medizin bis hin zur wissenschaftlichen Forschung. In der Industrie wird KI beispielsweise zur Automatisierung von Fertigungsprozessen und zur Optimierung von Lieferketten eingesetzt. In der Medizin hilft KI bei der Diagnose von Krankheiten durch die Analyse von medizinischen Bildern und Daten, während sie in der Forschung verwendet wird, um große Datensätze zu analysieren und neue Erkenntnisse zu gewinnen.\footcite[9]{gethmann_kunstliche_2022}

Künstliche Intelligenz (KI) findet in einer Vielzahl von Anwendungsbereichen Anwendung, darunter in der Industrie, in der Medizin sowie in der wissenschaftlichen Forschung. In der Industrie findet KI beispielsweise Anwendung in der Automatisierung von Fertigungsprozessen sowie der Optimierung von Lieferketten. Im medizinischen Kontext unterstützt KI die Diagnose von Krankheiten, indem sie medizinische Bilder und Daten analysiert. In der Forschung findet KI Anwendung bei der Analyse großer Datensätze, um neue Erkenntnisse zu gewinnen. Eine der grundlegenden Technologien, die KI antreibt, ist das maschinelle Lernen (ML). Maschinelles Lernen (ML) bezeichnet eine Gruppe von Algorithmen, die darauf abzielen, Systeme dazu zu befähigen, aus Daten zu lernen und ihre Leistung im Laufe der Zeit zu verbessern. Dabei werden Muster in Daten identifiziert und Vorhersagen getroffen. Ein bekanntes Beispiel für den Einsatz von ML ist die Verwendung in Empfehlungssystemen, wie sie bei Streaming-Diensten oder Online-Shops eingesetzt werden, um personalisierte Vorschläge zu generieren.\footcite[15]{oecd_oecd-bericht_2024}

\subsubsection{Natural Language Processing}

Natural Language Processing (NLP) stellt einen zentralen Bereich der Künstlichen Intelligenz (KI) dar, der sich mit der Interaktion zwischen Computern und menschlicher Sprache befasst. NLP-Technologien ermöglichen es Computern, Text zu verstehen, zu interpretieren und zu generieren. Dies bildet die Grundlage für Anwendungen wie Sprachassistenten, automatische Übersetzungen und Chatbots. Die Fähigkeit von Maschinen, natürliche Sprache zu verarbeiten, hat zu einer grundlegenden Veränderung der Art und Weise geführt, wie Menschen mit Computern interagieren.\footcite[22]{oecd_oecd-bericht_2024}

\section{KI-gestützte Assistenzsysteme}